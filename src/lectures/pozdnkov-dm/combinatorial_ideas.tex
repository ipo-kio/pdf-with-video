\documentclass[russian]{lecture-notes}

\usepackage[final]{graphicx}
\usepackage{subcaption}
\usepackage{timestamps}
\usepackage{sectsty}
\sectionfont{\clearpage}
\usepackage{hyperref}
\usepackage{float}
\usepackage{amsmath}
\usepackage{cancel}
\usepackage{amssymb}
\usepackage{tikz}
\usepackage{algorithm}
\usepackage{algpseudocode}
\usepackage{slashbox}

\renewcommand{\arraystretch}{1.3} %Для расстояния в таблицах
\cleartheorem{example*}
\theoremstyle{definition}
\newtheorem{example*}{Пример}[subsection]
\newtheorem*{exercise}{Упражнение}
\newtheorem*{solution}{Решение}
\newtheorem*{newProblem}{Составление задачи}
\newtheorem*{hats}{Задача о шляпах}
\newcommand{\divs}{\mathrel{\raisebox{-2pt}{\vdots}}}
\DeclareMathOperator{\Aa}{A}
\newcommand{\A}[2]{\Aa(#1;#2)}
\DeclareMathOperator{\Dd}{D}
\newcommand{\D}[1]{\Dd(#1)}
\newcommand{\DD}[2]{\Dd(#1;#2)}
\DeclareMathOperator{\Bb}{B}
\newcommand{\B}[1]{\Bb(#1)}
\DeclareMathOperator{\KRAZ}{D}
\newcommand{\DN}[1]{\KRAZ_{#1}}

\title{Лекция <<Замечательные идеи комбинаторики>>}
\lecturer{Поздняков Сергей Николаевич}
\notesauthor{Кацер Евгений}
\date{21 декабря 2018 г.}
\youtubevideo{ByNx9U3a0TU}

\begin{document}
	\maketitle
	
	\begin{center}
		\section*{\LARGE Глава II. Комбинаторика}
		\timestamp{00:04}
		\label{glav:komb}
	\end{center}

	Комбинаторика~--- одна из областей, вход в которую очень быстрый. То есть многие проблемы могут быть объяснены на пальцах и в то же время они могут быть до сих пор не решены. В комбинаторике очень интересно проводить компьютерный эксперимент, пытаться вывести какие-то общие формулы.
	
	Теперь немного о роли комбинаторики: как она связана с программированием, с курсом дискретной математики. Во-первых, в ней изучаются \emph{комбинаторные алгоритмы}. Например, перечислить все подмножества какого-то множества, или перебрать все перестановки, или перестановки, в которых часть элементов повторяется и так далее (эти генераторы и генерацию таких комбинаторных объектов мы будем разбирать через неделю). Во-вторых, в комбинаторике изучаются \emph{классические задачи} на подсчет числа объектов. 
	
	А что интересно считать в программировании?
	
	Ответ: \emph{трудоемкость алгоритмов}~--- сколько времени требуется на выполнение алгоритма, если размер входных данных $n$ (граф с $n$ вершинами, число с $n$ цифрами и др). Обратите внимание на классическую работу Дональда Кнута~--- <<Искусство программирования>>. В ней последний четвертый том посвящен комбинаторным задачам. 
	
	И еще хочется обратить ваше внимание на комбинаторное мышление, связанное с процессом изобретения. В свое время крупный математик Адамар интересовался тем, как рождаются новые открытия и провел анкету среди известных ученых, чтобы узнать, как появляется новое, как человек открывает то, что до него никогда не было. Оказывается, что процесс открытия связан с напряженной работой и актуализацией тех знаний которые у человека есть.
	
	У меня одна девушка решила по одной из тем альтернативного экзамена взять похожую тему, но попроще, чтобы с ней разобраться. Мы с ней уже два раза виделись, я ее спрашиваю: <<Ну как, вы разобрались?>> Она говорит: <<Нет, я хочу сама придумать этот алгоритм>>. Вот это правильный подход: надо сначала исчерпать все свои возможности по решению этой задачи. И вот тогда, когда эти возможности исчерпаны, подсознанием запускаются некие механизмы. В опросе Адамара ученые рассказывают, что происходило после такой упорной, но неуспешной деятельности: они бросали эту работу над задачей, кто-то ехал отдыхать, кто-то переключался на другую работу. И вдруг неожиданно всплывало решение этой задачи. Кто-то даже пытался последить за собой, как же это получается. В некоторых наблюдениях отмечалось, что у ученого в голове как бы сталкивались какие-то различные идеи, которые вдруг образовали хорошую комбинацию.
	
	Сейчас всех интересует искусственный интеллект, но если вы будете пытаться смоделировать какие-то интеллектуальные процессы, то поймете, что на каком-то уровне наступает, грубо говоря, перебор вариаций, и <<талант>> ученого заключается в выборе <<красивых>> комбинаций. И вот эта эстетическая оценка, которая формируется за счет того, что вы об этом много думаете, упорно работаете, все что можно собираете вокруг этой задачи, резко уменьшает объем комбинаций, потому что просто перебор требует времени не сопоставимого с продолжительностью жизни человека. Вы можете вспомнить про шахматы, где количество вариантов развития партии растет экспоненциально.
	
	Интересно привести пример рассуждения Галилея, позволившее ему <<переоткрыть>> телескоп. Он стал рассуждать, как может быть устроен телескоп: простые стекла там быть не могут, значит они выпуклые или вогнутые, если два выпуклых комбинировать, ничего не получится, если два вогнутых, тоже ничего не получится, значит надо выпуклое и вогнутое стекла комбинировать, и подобрать расстояние. Вот такими комбинаторными суждениями Галилей открыл для себя, как устроен телескоп.
	
	Комбинаторные задачи еще интересны тем, что каждая задача в комбинаторике предполагает свою модель. Их можно как-то классифицировать, собрать несколько похожих задач, но это как раз и интересно то, что вы можете к каждой задаче применять эти общие принципы и строить новую модель. Сейчас я и хочу продемонстрировать вам такие простые идеи. Когда-то меня попросили прочесть лекцию в Пакистане на английском языке, и я думал, что же прочитать, чтобы всем было интересно, и выбрал как раз комбинаторику. Прочитал лекцию, которая называется <<Brilliant ideas of combinatorics>>.
	
	\timestamp{08:37}
	
	Прежде чем изложить эти замечательные идеи, я еще раз обращусь к тому, насколько в комбинаторике просто поставить нерешенную задачу. Приведу пример задачи о меандрах, которую поставил Владимир Игоревич Арнольд: есть прямолинейная дорога и ее пересекает какая-то речка:
	\begin{figure}[H]
		\centering
		\tikz{
			\draw [thick] (-4,-2) -- (4,2);
			\draw [thick] (-3.5,-1) arc (180:120:0.5 and 0.25)
			arc (110:40:1.5 and 0.8) 
			arc (245:425:1)
			arc (65:230:0.3)
			arc (45:-50:1.5 and 0.7)
			arc (282:90:0.8 and 1.1)
			arc (85:20:1.75 and 1.5) -- (0.35, 0)
			arc (205:385:0.85) -- (1.6, 1.3)
			arc (200:45:0.3) -- (2.8, 0.8)
			arc (225:380:0.5)
			arc (20:100:1.75 and 1.5)
			arc (95:180:0.5 and 1) -- (1.3, 0.4)
			arc (5:-170:0.3)
			arc (190:90:1.75 and 2.25)
			arc (270:355:1 and 0.7);
			
			\coordinate [label=below:Дорога] (a) at (-3.8, -2);
			\coordinate [label=above:Река] (b) at (2, 3);
		}
		\caption{\small Пример дороги и реки}
	\end{figure}

	Вы можете мостики (пересечение дороги и реки) пронумеровать в порядке того, как вы проезжайте, двигаясь по дороге:
	\begin{figure}[H]
		\centering
		\tikz{
			\path [fill=black] (-1.825,-0.9) circle (0.75mm);
			\path [fill=black] (-1.485,-0.745) circle (0.75mm);
			\path [fill=black] (-0.66,-0.33) circle (0.75mm);  
			\path [fill=black] (-0.17,-0.09) circle (0.75mm);
			\path [fill=black] (0.29,0.14) circle (0.75mm); 
			\path [fill=black] (0.7,0.335) circle (0.75mm);
			\path [fill=black] (1.28,0.63) circle (0.75mm);
			\path [fill=black] (1.8,0.9) circle (0.75mm);
			\path [fill=black] (2.43,1.22) circle (0.75mm);
			\path [fill=black] (3.39,1.695) circle (0.75mm);
			
			\draw [thick] (-4,-2) -- (4,2);
			\draw [thick] (-3.5,-1) arc (180:120:0.5 and 0.25)
			arc (110:40:1.5 and 0.8) 
			arc (245:425:1)
			arc (65:230:0.3)
			arc (45:-50:1.5 and 0.7)
			arc (282:90:0.8 and 1.1)
			arc (85:20:1.75 and 1.5) -- (0.35, 0)
			arc (205:385:0.85) -- (1.6, 1.3)
			arc (200:45:0.3) -- (2.8, 0.8)
			arc (225:380:0.5)
			arc (20:100:1.75 and 1.5)
			arc (95:180:0.5 and 1) -- (1.3, 0.4)
			arc (5:-170:0.3)
			arc (190:90:1.75 and 2.25)
			arc (270:355:1 and 0.7);
			
			\coordinate [label=below:Дорога] (a) at (-3.8, -2);
			\coordinate [label=above:Река] (b) at (2, 3);
			
			\coordinate [label=below:1] (c) at (-1.825,-0.9);
			\coordinate [label=above:2] (d) at (-1.485,-0.745);
			\coordinate [label=-45:3] (e) at (-0.73,-0.29);  
			\coordinate [label=135:4] (f) at (-0.1,-0.2);
			\coordinate [label=below:5] (g) at (0.29,0.14); 
			\coordinate [label=45:6] (h) at (0.6,0.335);
			\coordinate [label=-45:7] (i) at (1.28,0.63);
			\coordinate [label=above:8] (j) at (1.8,0.9);
			\coordinate [label=above:9] (k) at (2.43,1.22);
			\coordinate [label=above:10] (l) at (3.39,1.695);
		}
		\caption{\small Пример дороги и реки}
	\end{figure}

	Получилось 1, 2, 3, 4, 5, 6, 7, 8, 9, 10. Далее можно плыть по этой речке в том же направлении и смотреть в каком порядке проходят мостики. Получится 1, 4, 3, 2, 5, 8, 9, 10, 7, 6. Вот у нас получилась перестановка 10 чисел. Понятно, что не любую перестановку можно изобразить \emph{меандром} (меандр~--- форма реки). Так вот эта задача о количестве меандров. У меня как-то несколько лет назад два студента независимо взялись решать эту задачку численно, и в какой-то момент показалось, что они поставили мировой рекорд (на то время). Количество меандров быстро возрастает, студенты дошли до 14 и 15, а дальше уже не смогли. Тогда они стали изучать литературу более внимательно, нашли, что уже кто-то их обогнал, и 14 и 15 не были лучшим результатом, но важно, что хороших результатов они добились сами.
	
	Видите, задачка очень понятно поставлена и до сих пор не решена, то есть нет формулы, есть оценки сверху и снизу для количества меандров. Я думаю, что у Арнольда есть и другие интересные комбинаторные задачи.
	
	\timestamp{12:04}
	
	Перейдем к замечательным идеям комбинаторики:
	\begin{enumerate}
		\item \textbf{Правило умножения}. Вы знаете, что и в алгебре, и в программировании есть двумерные массивы, матрицы. И матрицы естественно использовать для того, чтобы классифицировать какие-то объекты по 2 параметрам. Допустим у вас есть множество букв~--- $\{a, b, c\}$ и есть множество чисел~--- $\{1, 2, 3, 4\}$, тогда в таблице естественным образом можно перечислить их комбинации (точнее пары составленные из букв и цифр):
		
		\begin{table}[H]
			\centering
			\begin{tabular}{|c|c|c|c|c|}
				\cline{2-5}
				\multicolumn{1}{c|}{} & 1 & 2 & 3 & 4 \\ \hline
				$a$ & $a1$ & $a2$ & $a3$ & $a4$ \\ \hline
				$b$ & $b1$ & $b2$ & $b3$ & $b4$ \\ \hline
				$c$ & $c1$ & $c2$ & $c3$ & $c4$ \\ \hline
			\end{tabular}
		\end{table}
	
		Вы знаете, наверное, что такая конструкция называется <<Декартово произведение двух множеств>>. То есть, если множество букв обозначить множеством А от слова Alphabet, а множество цифр~--- D от слова Digit, то получится что эта таблица есть не что иное, как произведение этих двух множеств:
		
		\[
			A = \{a, b, c\}; \ D = \{1, 2, 3, 4\}
		\]
		\[
			A \times D = \{(a_i, d_i) | a_i \in A, d_i \in D\}
		\]
		
		\noindent Так определяется декартово произведение. Понятно, что если вам нужно посчитать количество элементов этого множества $|A \times D|$... (Эти палочки ($| \ |$) в комбинаторике обозначают мощность множества или количество элементов, если это множество конечно). С понятием мощности вы скорее всего уже сталкивались. Мощность обладает интересными свойствами для бесконечных множеств. Для конечных все проще: если мы умножим количество элементов $A$ на количество элементов $D$, то мы получим сколько элементов произведения:
		
		\[
			|A \times D| = |A| \cdot |D|
		\]
		
		\timestamp{15:17}
		
		А вот, когда у нас множество бесконечное, появляется очень любопытный эффект. Так математик и популяризатор~--- Наум Яковлевич Виленкин, который, кстати, написал книжку <<Комбинаторика>>, в другой книге <<Рассказы о множествах>> эффекты с мощностью бесконечных множеств облек в фабулу одного из произведений Станислава Лема~--- <<Звездные дневники Ийона Тихого>>. Фабула такая: есть отель в котором бесконечное количество номеров, и вот туда прилетает Ийон Тихий, а все номера заняты. Но администратор поступает очень просто: он жильца из первого номера переселяется во второй, из второго в третий, из третьего в четвёртый, а Ийона Тихого помещают в первый. После него прилетает делегация из ста человек. Понятно, что нужно делать: надо первого в сто первый, второго в сто второй, а делегацию поселить на освободившиеся места. Но потом прилетает делегация из бесконечного количества человек, тогда администратор селит первого во второй, второго в четвертый, третьего в шестой, K-ого в 2K-ый. Все нечетные номера освобождаются, и администратор селит в них бесконечную делегацию. Но после этого прилетает бесконечное число бесконечных делегаций. Что делать тогда? Тогда администратор строит похожую на нашу табличку, в которой строка~--- номер делегации, а столбец~--- номер члена делегации. Только теперь эта таблица не конечная, а бесконечная:
		
		\begin{table}[H]
			\centering
			\caption{\small Таблица соответствует каждому члену делегации}
			\begin{tabular}{|c|c|c|c|c|}
				\hline
				\backslashbox{№ дел.}{№ чл.} & 1 & 2 & 3 & \ldots \\ \hline
				$1$ &  &  &  & \ldots \\ \hline
				$2$ &  &  &  & \ldots \\ \hline
				$3$ &  &  &  & \ldots \\ \hline
				\vdots & \vdots & \vdots & \vdots & $\ddots$
			\end{tabular}
		\end{table}
	
		\noindent Давайте теперь расселим всех членов делегации. Селить их будем по <<диагоналям>> таблицы:
		
		\begin{table}[H]
			\centering
			\caption{\small Таблица соответствует каждому члену делегации}
			\begin{tabular}{|c|c|c|c|c|}
				\hline
				\backslashbox{№ дел.}{№ чл.} & 1 & 2 & 3 & \ldots \\ \hline
				$1$ & №1 & №2 & №4 & \ldots \\ \hline
				$2$ & №3 & №5 &  & \ldots \\ \hline
				$3$ & №6 &  &  & \ldots \\ \hline
				\vdots & \vdots & \vdots & \vdots & $\ddots$
			\end{tabular}
		\end{table}
	
		\noindent Заметим, что это не единственный способ расселения.
		
		\timestamp{18:22}
		
		\begin{exercise}
			В какой номер будет поселен $j$-ый член $i$-ой делегации?
		\end{exercise}
	
		Заметим, что формулы для генерации нам понадобятся, поэтому это упражнение весьма полезно. Пример с <<бесконечной>> гостиницей показывает, почему естественно считать, что четных чисел столько же, сколько натуральных, почему рациональных чисел столько же, сколько целых. Мы установили взаимно-однозначное соответствие между элементами множеств, поэтому и считаем множества равномощными.
		
		Почему вещественных чисел больше чем рациональных я не буду рассказывать. Это к сегодняшней лекции отношения не имеет. Вот через полгода, когда мы будем с вами обсуждать теорию алгоритмов, это будет очень важно, потому что как раз есть задача, которых множество задач равномощно множеству вещественных чисел, а множество алгоритмов эквивалентно множеству натуральных чисел, и поэтому понятно, что есть задачи для которых нет алгоритма решения. Применим принцип умножения к простой задачке. 
		
		\timestamp{20:34}
		
		\begin{problem}
			Сколько существует различных автомобильных номеров? 
		\end{problem}
	
		\begin{solution}
			Как устроен номер? Сначала идет буква, потом три цифры, потом две буквы. То есть множество номеров можно записать в таком виде:
			
			\[
				A \times D \times D \times D \times A \times A
			\]
			
			Множества $A$ и $D$ будут больше, чем в предыдущем примере. Теперь, если мне нужно найти количество этих номеров, я могу воспользоваться той же самой идеей. Ну понятно, что у меня будет уже таблица не квадратная \fbox{$A \times D$}$\times D \times D \times A \times A$), не кубическая (\fbox{$A \times D \times D$}$\times D \times A \times A$), не четырехмерно кубическая (\fbox{$A \times D \times D \times D$}$\times A \times A$), не пятимерно кубическая (\fbox{$A \times D \times D \times D \times A$}$\times A$), а будет шестимерный куб. Точнее не куб, а параллелепипед, потому что не все стороны равны. Но все равно его объем, в данном случае, это количество элементов, считается по такой же формуле:
			
			\[
				|A \times D \times D \times D \times A \times A| = |A| \cdot |D| \cdot |D| \cdot |D| \cdot |A| \cdot |A|
			\]
			
			В правой части выражения мы перемножаем числа. В левой части выражения множества менять местами нельзя, иначе получатся другие наборы, а вот в правой $|A|$~--- количество букв, $|D|$~--- количество цифр, и множители можно переставлять:
			
			\[
			|A| \cdot |D| \cdot |D| \cdot |D| \cdot |A| \cdot |A| = |A|^3 \cdot |B|^3
			\]
			
			Если использовать около 20 букв и 10 цифр, то количество номеров легко оценить:
			
			\[
			|A|^3 \cdot |B|^3 \approx 20^3 \cdot 10^3 = 8000000
			\]
		\end{solution}
		
		Следующим будет правило сложения или принцип сложения.
		
		\timestamp{23:18}
		
		\item \textbf{Принцип сложения}. Вы уже слышали, наверно, есть такой подход к решению задач <<Divide and conquer>>~--- <<Разделяй и властвуй>>. Вот, в частности, принцип сложения в комбинаторике относится к той же идее. Если вы не можете решить трудную задачу, разбейте ее на части и решайте эти части. Но на самом деле этот процесс может быть итеративный, и у нас будут получаться так называемые рекуррентные формулы, когда мы не можем написать сразу, как надо делать вычисления, но каждый шаг, переход от одного к другому, мы можем записать. Продемонстрируем принцип сложения на задаче размена.
		
		\timestamp{24:04}
		
		\begin{problem}
			Сколькими способами можно разменять 1 рубль монетами достоинством: 50 копеек, 10 копеек, 5 копеек.
		\end{problem}
			
		\begin{solution}
			Понятно, что такое размен монет: порядок в разменном наборе роли не играет, например, если вам сначала дали 50 копеек, а потом пять раз по 10 копеек, то это то же самое, если вам сначала дадут 10, потом 50, а потом еще четыре монеты по 10, то это то же самое. То есть нас не интересует порядок. Нас интересует набор монет, только из каких монет он составлен. А в каком порядке нам эти монеты выдали, мы учитывать не будем. Давайте обозначим $\DN{100}$~--- количество разменов одного рубля (рубль равен 100 копейкам). И вот теперь все способы мы разобьем на два типа: есть в них 50 копеечная монета или нет. Значит, если 50 копеечная монета есть, то нам остается разменять 50 копеек:
			
			\[
				\DN{100} = \DN{50} + \ldots
			\]
			
			Теперь рассмотрим случай когда в наборе нет 50 копеечных монет. Значит мне нужно разменять рубль:
			
			\[
				\DN{100} = \DN{50} + \DN{100}', \text{ где}
			\]
			
			 $\DN{100}'$~--- количество разменов 1 рубля монетами достоинством 10 копеек и 5 копеек. Согласны с тем, что я правильно разложил общее число вариантов в сумму двух чисел? Значит еще раз: если 50 копеек есть, то мне осталось разменять 50, то есть я размениваю 50 копеек, потом каждому набору добавляю еще 50 копеечную монету. Либо 50 копеечной монеты в наборе нет, и я свожу задачу к более простой, где меньше вариантов выбора разменных монет. Теперь рассмотрим размен 50 копеечной монеты. Опять два варианта: либо там есть 50 копеечная монета, и тогда разменивать больше нечего (остается разменять ноль), либо надо разменять 50 копеек более мелкими:
			 
			 \[
			 	\DN{100} = \DN{50} + \DN{100}' = \DN{0} + \ldots
			 \]
			 
			 Понятно какое число $\DN{0}$ естественно присвоить~--- единицу, потому что это означает, что рубль разменяли одним таким способом~--- две монеты по 50 копеек.
			 
			 \[
			 	\DN{100} = \DN{50} + \DN{100}' = (\underbrace{\DN{0}}_{1} + \DN{50}') + \ldots
			 \]
			 
			 Дальше, соответственно, разложим $\DN{100}'$. Либо там есть десятикопеечная монета, либо там нет 10 копеечных монет. Если десятикопеечная монета есть, нам остается разменять 90, если десятикопеечной монеты нет, то размениваем рубль (100 копеечную монету), не используя ни 50-копеечиные, ни 10-копеечные монеты:
			 
			 \[
			 	\DN{100} = \DN{50} + \DN{100}' = (\underbrace{\DN{0}}_{1} + \DN{50}') + (\DN{90} + \DN{100}''), \text{ где}
			 \]
			 
			 $\DN{100}''$~--- количество разменов 1 рубля монетами по 5 копеек. Сколько таких вариантов? Один! Это двадцать монет, но размен всего один. Значит после этого получается:
			 
			 \[
				 \DN{100} = \DN{50} + \DN{100}' = (\underbrace{\DN{0}}_{1} + \DN{50}') + (\DN{90} + \underbrace{\DN{100}''}_{1}) = 2 + \DN{50}' + \DN{90}'
			 \]
			 
			 Теперь раскладываем аналогично $\DN{50}'$. Либо в наборе есть 10 копеек, либо нет. Если есть, остается 40, либо 50 размениваем монетами по 5 копеек. Опять же для этого будет один способ. То же самое $\DN{90}'$.
			 
			 \[
			 	2 + \DN{50}' + \DN{90}' = 2 + (\DN{40}' + \underbrace{\DN{50}''}_{1}) + (\DN{80}' + \underbrace{\DN{90}''}_{1}) = \ldots
			 \]
			 
			 И вот теперь, наверное, можно поставить многоточие и сообразить сколько шагов мы так сделали. Значит было 50, потом стало 40 и еще добавилась единичка, 30 еще единичка 20, 10, 0, да? Посчитали сколько там добавятся единичек или не посчитали? Еще 4, наверное. То есть $\DN{40}' + \DN{50}'' = 4 + 1 = 5$ [скорее всего здесь 3 + 1, так как в дальнейшем мы пишем $\DN{0}' + \DN{10}''$] и плюс еще 2 в начале $=7$. И еще за эти шаги $\DN{80}' + \DN{90}''$ добавит 5.
			 
			 \[
			 	2 + (\DN{40}' + \underbrace{\DN{50}''}_{1}) + (\DN{80}' + \underbrace{\DN{90}''}_{1}) = 12 + \underbrace{\DN{0}'}_{1} + \underbrace{\DN{10}''}_{1} + \DN{40}' + \underbrace{\DN{50}''}_{1} =
			 \]
			 \[
			 	= 15 + \DN{40}'
			 \]
			 
			 $\DN{40}'$ даст нам еще 5, получится:
			 
			 \[
			 	15 + \DN{40}' = 20
			 \]
			 
			 Мне кажется, что я где-то ошибся, чего-то у меня ощущение, что их то ли 18, то ли 19. На самом деле $\DN{40}' = 4$, поэтому:
			 
			 \[
			 	15 + \DN{40}' = 19
			 \]
			 
			 Я предлагаю вам просто перебрать все эти наборы после лекции и проверить, что мы нигде в вычислениях не ошиблись.
			
		\end{solution}
	
		Формулы подобные той, которую мы получили или получали, называют рекуррентными. Подобные задачи называются задачами на разбиение числа на слагаемые.
		
		\timestamp{34:31}
		
		\begin{problem}
			Обобщить предыдущую задачу на случай размена не одного рубля, а N рублей.
		\end{problem}
	
		Теперь я хочу вам показать один прием, который позволяет нам использовать разные способы рассуждения для того, чтобы вывести различные комбинаторные свойства и формулы.
		
		\timestamp{35:27}
	
		\item \textbf{Сравнение различных способов решения}. Вы знаете уже, что когда мы применили теорему Безу к задаче интерполяции, то мы нашли связь между алгеброй и анализом. А сейчас мы будем искать такие связи внутри комбинаторики.
		
		\begin{problem}
			Требуется из N человек выбрать команду и в этой команде выбрать капитана. Сколькими способами это можно сделать?
		\end{problem}
	
		\begin{solution}
			 Я не знаю все ли проходили в школе, но есть такой замечательный объект $C_n^k$~--- число способов выбора $k$ предметов из $n$ различных. То есть, если я хочу выбрать из вас команду из 10 человек, а вас допустим 120, это будет $C_{120}^{10}$. Я мог бы это перевести на более формальный математический язык и сказать, что $C_n^k$~--- число $k$-элементных подмножеств множества из $n$ элементов. Теперь будем считать, что вы не проходили, что такое $C_n^k$ и вы слышите впервые, что я сказал, и не знаете никакой формулы, как $C_n^k$ вычисляется. Можем ли мы <<из ничего>> (точнее из определения) извлечь формулу для вычисления? Давайте вернемся к принципу умножения и к той задаче, которую я вам только что поставил. Допустим мы действуем так: сначала выберем команду из $k$ человек, а потом в этой команде выберем капитана. Сколькими способами можно выбрать капитана? $k$ человек в команде. Любой человек в команде может быть капитаном. Понятно, что $k$ и $C_n^k$~--- независимые комбинации, да? Поэтому я могу перемножить:
			 
			 \[
			 	kC_n^k
			 \]
			 
			 Могу ли я рассуждать по-другому? Могу сначала выбрать капитана вообще из всех. Сколькими способами я могу выбрать капитана? N способами. А потом к нему добавить команду, но раз я уже капитана выбрал, он уже не считается. Значит у меня осталось $n-1$ человека. И сколько мне надо выбрать из них? $k-1$:
			 
			 \[
			 	nC_{n-1}^{k-1}
			 \]
			 
			 Но мы решали одну и туже задачу, поэтому и ответ должен получиться тот же самый:
			 
			 \[
			 	kC_n^k = nC_{n-1}^{k-1}
			 \]
			 
			 А теперь разделим обе части на $k$:
			 
			 \[
			 	C_n^k = \frac{n}{k}C_{n-1}^{k-1}
			 \]
			 
			 Видите, мы получили рекуррентную формулу, которая позволяет нам свести задачу к более простой. Мы всегда можем уменьшать параметры до того, пока $k-1$ не станет равным 1, и задача станет очевидной. Например, сколькими способами из 100 человек можно выбрать одного? 100. 
			 
			 \[
			 	C_n^1 = n
			 \] 
			 
			 Поэтому достаточно такой простой формулы, чтобы потом вычислить любое число сочетаний. Выведем нерекуррентную формулу. Для этого можно и немножко продолжить начатый процесс. Тогда получится:
			 
			 \[
			 	C_n^k = \frac{n}{k} \cdot C_{n-1}^{k-1} = \frac{n}{k} \cdot \frac{n-1}{k-1} \cdot C_{n-2}^{k-2}
			 \]
			 
			 И так мы будем идти дальше. По смыслу $k$ всегда не будет превышать $n$: я же не могу из ста человек выбрать 200. Это будет 0, поэтому все разумные случаи предполагают, что $k$ не больше $n$, поэтому $k$ кончится раньше, и если я напишу многоточие в выражении, то конец у меня будет вот такой:
			 
			 \[
			 	C_n^k = \frac{n}{k} \cdot \frac{n-1}{k-1} \cdot \ldots \cdot \frac{}{1}
			 \]
			 
			 В последней дроби внизу у меня будет единичка, а что будет наверху? \underline{Ответ из зала}: $n-k$. Все согласны? А вот смотрите, как легко проверить: разница между числителем и знаменателем в первой дроби $n - k$, во второй естественно та же самая, мы же вычли по единичке из числителя и из знаменателя, значит и в конце эта разница должна быть той же самой.  Это инвариант цикла. То, о чем мы уже говорили. В последней дроби какой инвариант цикла? $n - k - 1$, значит наверху... То есть инвариант цикла $n - k$, а наверху что должно стоять? $n - k + 1$, чтобы после вычитания из числителя знаменателя получилось $n - k$:
			 
			 \[
			 	C_n^k = \frac{n}{k} \cdot \frac{n-1}{k-1} \cdot \ldots \cdot \frac{n - k + 1}{1}
			 \]
			 
			 Получившееся выражение в западных учебниках обозначают вот так:
			 
			 \[
			 	\left(
			 	\begin{array}{c}
				 	n \\
				 	k
			 	\end{array}
			 	\right)
			 \]
			 
			 Понятно почему: берем $\frac{n}{k}$, а потом от числителя и знаменателя вычитаем по единичке, пока это сохраняет смысл и перемножаем дроби. Поэтому, если вы будете читать Кнута, например, то вы увидите вместо $C_n^k$ вот такое обозначение. Мне почему-то приятнее работать с $C_n^k$. Может быть, потому что я в детстве прочитал Виленкина и на всю жизнь запомнил. А может потому, что западное обозначение не похоже на число. Но полезно, конечно, пользоваться любыми обозначениями.
			 
		\end{solution}
	
		Следующая задача связана с самым главным, наверное, приемом в комбинаторике~--- это взаимно-однозначное соответствие.
		
		\timestamp{42:59}
		
		\item \textbf{Взаимно-однозначное соответствие}.Я вам задам задачку. Есть такой журнал <<Квант>>, до сих пор существующий, а появился он в 1970 году. Тогда количество людей, интересовавшихся математикой и читавших журнал, было таким, что этот журнал был самым известным в мире~--- лучшим журналом по математике для школьников. Так вот я вам предложу задачу из рубрики <<Кванта для младших школьников>>. Вы знаете, что такое счастливый билет? Трамвайный. Например, 127541, потому что сумма первых трех цифр равняется 10 и сумма последних трех цифр тоже 10. Так вот оказывается, что счастливых билетов столько же, сколько билетов суммой цифр 27:
		
		\[
			\underbrace{239436}_{\sum = 27}
		\]
		
		\timestamp{45:32}
		
		\begin{problem}
			Вам нужно придумать взаимно-однозначное соответствие, чтобы каждому счастливому билету сопоставлялся билет суммой цифр 27 и наоборот.
		\end{problem}
		
		\begin{solution}
			А теперь перейдем к этой задаче есть у кого-нибудь идея или мне показать решение? Идея вот какая: одну из троек цифр, первую или вторую, дополним до 9, то есть вместо этих трех цифр (обведены в квадрат), мы напишем разности девяток с ними:
			
			\[
				\begin{array}{cc}
				& 999 \\
				127 & \fbox{541} \\
				& 458
				\end{array} \longleftrightarrow \underbrace{127458}_{\sum = 27}
			\]
			
			Получившийся номер будет иметь сумму цифр 27. Согласны? Ну, понятно, смотрите, суммы цифр $541$ с $458$ дает три девятки, но суммы цифр $127$ и $541$ одинаковы, поэтому $127$ с $458$ дадут ту же сумму цифр. Это правило работает в обес стороны, поэтому мы установили взаимное однозначное соответствие между <<счастливыми билетами>> и шестизначными наборами с суммой цифр 27.
		\end{solution}
		
		Рассмотрим еще один интересный принцип или интересную идею, которая связана с различными языками описания комбинаторных объектов.
		
		\timestamp{47:47}
		
		\item \textbf{Различные языки описания комбинаторных объектов}. Если вы будете читать разные учебники, то заметите, что авторы стараются изложить какую-то новую идею на основе предыдущей, поэтому они выбирают один язык представления математических объектов и стараются все математические идеи изложить на нем. На самом деле это для комбинаторики не всегда удобно: иногда чуть-чуть заменишь язык и все становится понятным, прозрачным. Даже есть такая максима, что любое утверждение можно сделать понятным, если его правильно переформулировать. Так что один из способов в чем-то разобраться~--- найти способ представления новой идеи, который сделал бы результат очевидным.
		
		Давайте рассмотрим такую задачу: допустим празднуются чей-то день рождения и одного человека посылают за пирожными. В магазине 5 видов пирожных, каждого из которых достаточно много, чтобы можно было купить все пирожные одного вида. Нам нужно купить 10.
		
		\timestamp{49:26}
		
		\begin{problem}
			Сколькими способами можно купить 10 пирожных, если имеется 5 сортов, а количество пирожных каждого сорта неограниченно.
		\end{problem}
	
		Если попробуете перечислять такие наборы, то окажется, что это довольно трудно. Правило умножения здесь не будет работать, потому что порядок не важен. Вот мы купили, например, 10 эклеров, другой вариант~--- 9 эклеров и какой-нибудь буше, как тут правило перемножения работает? Если бы каждые пирожные покупались независимо, тогда, да. У каждого пирожного было бы пять способов, вы бы эту 10 раз перемножили пятерку, получили бы $5^{10}$ и сказали: <<Вот столько способов>>. Но потом бы вам кто-нибудь сказал: <<Хорошо, вот вы взяли первое пирожное эклер, второе~--- буше, а теперь наоборот~--- первое буше, а второе эклер. Это что у вас разные способы получились?>> Вы бы сказали:<< Ай-ай-ай, я видимо очень много раз посчитал одно и то же>> и стали бы думать как это исправить. Если бы в этом множестве каждый набор повторялся одинаковое число раз, то мы бы просто поделили $5^{10}$ на число повторений (как это делается при выводе формулы размещений через перестановки) но здесь ситуация иная. Я вам покажу способ поиска другого языка.
		
		\begin{solution}
			Сейчас мы придумаем код для пирожных: $\oslash$~--- так будем обозначать пирожное, можно считать, что это нолик. А вот так обозначим разделитель между разными сортами пирожных $|$, можно считать, что это единица. Например, $\oslash \oslash |$ так изобразим 2 пирожных первого сорта (пусть это будут эклеры) с палочкой, отделяющей их от остальных сортов.
			
			\[
				\underbrace{\oslash \oslash}_{\text{эклер}} |
			\]
			
			А может стоять еще одна <<палочка>> $|$ сразу за $\oslash \oslash |$. Что это значит? Это значит, что буше, которое мы считаем вторым по счету сортов, отсутствует в нашей покупке:
			
			\[
				\underbrace{\oslash \oslash}_{\text{эклер}}  \underbrace{| \ \ \ \ |}_{\text{буше}}
			\]
			
			Какие вы еще пирожные знаете? \underline{Ответ из зала}: берлинер. О, я даже такого не знаю:
			
			\[
				\underbrace{\oslash \oslash}_{\text{эклер}}  \underbrace{| \ \ \ \ |}_{\text{буше}} \underbrace{\oslash \oslash \oslash}_{\text{берлинер}}
			\]
			
			Так, хорошо, потом сколько у нас сортов то всего... Пять: раз, два, три. Какой ещё сорт назовете? \underline{Ответ из зала}: тарталетка. Современные студенты знают о пирожных больше предыдущего курса, там они вообще, кроме булочки с маком, ничего не знали:
			
			\[
				\underbrace{\oslash \oslash}_{\text{эклер}}  \underbrace{| \ \ \ \ |}_{\text{буше}} \underbrace{\oslash \oslash \oslash}_{\text{берлинер}} | \underbrace{\oslash}_{\text{тарталетка}}
			\]
			
			Хорошо, значит у нас раз, два, три, четыре сорта и еще пятый. Четыре палочки уже поставил, значит один сорт пирожного остался. Ну, еще напрягитесь, какие еще бывают? \underline{Ответ из зала}: картошка. Хорошо, это хоть я знаю. И сколько пирожных у нас осталось? 4? Чтобы 10 получилось.
			
			\[
				\underbrace{\oslash \oslash}_{\text{эклер}}  \underbrace{| \ \ \ \ |}_{\text{буше}} \underbrace{\oslash \oslash \oslash}_{\text{берлинер}} | \underbrace{\oslash}_{\text{тарталетка}} | \underbrace{\oslash \oslash \oslash \oslash}_{\text{картошка}}
			\]
			
			Ну вот, согласны, что это один из видов покупки, да? И вот я теперь перейду на язык двоичных наборов: вместо пирожных я поставлю нули, а вместо границ между сортами я поставлю единички:
			
			\[
				\underbrace{\oslash \oslash}_{\text{эклер}}  \underbrace{| \ \ \ \ |}_{\text{буше}} \underbrace{\oslash \oslash \oslash}_{\text{берлинер}} | \underbrace{\oslash}_{\text{тарталетка}} | \underbrace{\oslash \oslash \oslash \oslash}_{\text{картошка}} \longleftrightarrow 0011000101000
			\]
			
			И получается, что вот этой покупке я присвоил вот такой двоичный код. Но при этом мы договорились, что порядок пирожных он всегда сохранится. Эклер~--- это сорт номер один, буше~--- сорт номер два, берлинер~--- сорт номер три, тарталетка~--- сорт номер четыре, картошка~--- сорт номер пять, иначе, конечно, не расшифровать. Ну, давайте я возьму любой другой набор. Из скольки единичек? Из четырех и скольких нулей? Из десяти. Почему единичек 4? Потому что они делят все на 5 частей. Например:
			
			\[
				1 1 0 0 0 1 0 0 0 0 1 0 0 0
			\]
			
			Что это за покупка? Ноль эклеров, ноль буше, три берлинера, четыре тарталетки и три картошки. Так что думаю, все согласятся, что это взаимно-однозначное соответствие.
			
			\timestamp{55:52}
			
			А теперь задачу ставлю так: сколько существует бинарных наборов (двоичных наборов) из четырех единиц и 10 нулей.
			
			Решаем так же, как решали задачу о команде с капитаном. Всего есть 14 позиций для цифр, то есть имеем 14-значный набор. Я выбираю 4 позиции и ставлю на них единички, а на остальные ставлю 0, и получается? $C_{14}^4$. Могу выбрать позиции для нулей
			
			\[
				C_{14}^4 = C_{14}^{10}
			\]
			
		\end{solution}
	
		Обратите внимание, что опять <<бесплатно>> получили формулу, которую вы, наверное, знаете:
		
		\[
		C_n^k = C_{n-k}^k
		\]
		
		Когда вы $k$ человек выбрали, $n - k$ осталось. Вместо того чтобы выбирать, кого я беру в команду, я могу выбирать, кого не беру. Это то же самое.
		
		Кстати сказать, тут могут быть интересные вариации: допустим я предлагаю купить так, чтобы каждого пирожного было бы хотя бы по единичке. Вы можете сказать, что эту задачу мы можем свести к предыдущей. Раз каждого пирожного по единичке мы эти пять пирожных отложим, а потом остальные пять выберем по той же самой формуле. Но можно рассуждать и в лоб, можно сразу попробовать придумать двоичный набор. Вот у нас 10 пирожных:
		
		\[
			\oslash . \oslash . \oslash . \oslash . \oslash . \oslash . \oslash . \oslash . \oslash . \oslash
		\]
		
		И теперь поскольку я знаю, что хотя бы одно пирожное каждого сорта у меня должно быть в наборе, я должен поставить между ними на 4 выбранных мною места (где точки) палочки. То есть теперь две палочки подряд идти не могут как раньше:
		
		\[
			\oslash . \oslash . \oslash | \oslash | \oslash . \oslash | \oslash . \oslash . \oslash | \oslash
		\]
		
		Вот четыре палочки поставили, что мы купили? Три эклера, одно буше и так далее... Сколькими способами это можно сделать? Из этих 9 точечек я должен выбрать 4, на которые я должен поставить разделитель~--- $C_9^4$. Вот, пожалуйста вам и другой способ решения (другой, правда) задачи, но прием аналогичный, когда мы придумываем новый язык двоичных кодов.
		
		\timestamp{59:33}
		
		И последнее: хочу вам тоже прорекламировать другую книгу Кнута. Она называется~--- "Конкретная математика". Там три автора (Грэхэм, Поташник и Кнут), и по ней тоже можно взять альтернативный экзамен. В ней есть ещё одна идея, которую Кнут очень красиво в книжке обыгрывает. Он рассматривает цепочки из домино и работает с ними как с алгебраическими объектами.
		
		\timestamp{60:22}
		
		\item \textbf{От комбинаторики перейти к алгебре}. Домино так устроено, что у него высота в два раза больше ширины:
		
		\begin{figure}[H]
			\centering
			\tikz{
				\draw [thick] (-1.75, 0) rectangle (-1.25, 1);
				\draw [thick] (-1.25, 1) rectangle (-0.25, 0.5);
				\draw [thick] (-1.25, 0.5) rectangle (-0.25, 0);
				\draw [thick] (-0.25, 1) rectangle (0.75, 0.5);
				\draw [thick] (-0.25, 0.5) rectangle (0.75, 0);
				\draw [thick] (0.75, 0) rectangle (1.25, 1);
				\draw [thick] (1.25, 0) rectangle (1.75, 1);
				
				\draw [fill = black] (1.9, 0.04) circle (0.5mm);
				\draw [fill = black] (2.05, 0.04) circle (0.5mm);
				\draw [fill = black] (2.2, 0.04) circle (0.5mm);
			}
			\caption{\small Пример последовательности доминошек}
		\end{figure}
		
		Можно, строя полосочки, класть либо две штучки горизонтально, либо одну вертикально. Мы хотим изучить множество таких цепочек. Кнут не вводит никаких переменных, а прямо начинает с этими цепочками совершать алгебраические операции. То есть он вводит операции прямо на множестве таких картинок. Я могу попробовать пояснить почему он решил поиграть с картинками полосок. Может быть, вы слышали о такой системе верстки математических текстов под называнием TeX. Так вот Кнут автор не только вот книги <<Искусство программирования>>, но он также создал систему верстки TeX и он здесь показывает, что в его системе он вместо букв может оперировать костяшками домино, например. И вот что он предлагает: <<Давайте мы все эти цепочки разделим на две части>>. Цепочка начинаться с вертикальной костяшки, и тогда символ умножения означает что я ее прикладываю. Эта операция в дальнейшем будет называться будет конкатенацией, когда мы будем изучать формальные языки. И тогда цепочку можно записать с использованием алгебраической операции вот так:
		
		\begin{figure}[H]
			\centering
			\tikz{
				\draw [thick] (0, 0) rectangle (0.5, 1);
				\draw [thick] (0.5, 1) rectangle (1.5, 0.5);
				\draw [thick] (0.5, 0.5) rectangle (1.5, 0);
				\draw [thick] (1.5, 1) rectangle (2.5, 0.5);
				\draw [thick] (1.5, 0.5) rectangle (2.5, 0);
				\draw [thick] (2.5, 0) rectangle (3, 1);
				\draw [thick] (3, 0) rectangle (3.5, 1);
				
				\draw [fill = black] (3.65, 0.04) circle (0.5mm);
				\draw [fill = black] (3.80, 0.04) circle (0.5mm);
				\draw [fill = black] (3.95, 0.04) circle (0.5mm);
				
				\draw [thick] (4.1, 0.525) -- (4.45, 0.525);
				\draw [thick] (4.1, 0.475) -- (4.45, 0.475);
				
				\draw [thick] (4.6, 0) rectangle (5.1, 1);
				\draw [fill = black] (5.25, 0.475) circle (0.5mm);
				
				\draw [thick] (5.65, -0.2) arc (-90:-270:0.25 and 0.7);
				
				\draw [thick] (5.75, 1) rectangle (6.75, 0.5);
				\draw [thick] (5.75, 0.5) rectangle (6.75, 0);
				\draw [thick] (6.75, 1) rectangle (7.75, 0.5);
				\draw [thick] (6.75, 0.5) rectangle (7.75, 0);
				\draw [thick] (7.75, 0) rectangle (8.35, 1);
				\draw [thick] (8.35, 0) rectangle (8.75, 1);
				
				\draw [fill = black] (8.9, 0.04) circle (0.5mm);
				\draw [fill = black] (9.05, 0.04) circle (0.5mm);
				\draw [fill = black] (9.2, 0.04) circle (0.5mm);
				
				\draw [thick] (9.4, -0.2) arc (-90:90:0.25 and 0.7);
			}
			\caption{\small Пример умножения доминошек}
		\end{figure}
	
		Рассмотрим множество всех цепочек. Разобъем это множество на два подмножества, используя для этого знак <<плюс>>~--- подмножество цепочек, начинающихся с вертикальной доминошки и подмножество, начинающихся с двух горизонтальных. Каждое из подмножеств представим как произведение:
		
		\begin{figure}[H]
			\centering
			\tikz{
				\coordinate [label = below:Мн-во всех] (A) at (2.45, 0.9);
				\coordinate [label = below:домино-цепочек] (A1) at (2.45, 0.4);
				
				\draw [thick] (4.4, 0.525) -- (4.65, 0.525);
				\draw [thick] (4.4, 0.475) -- (4.65, 0.475);
				
				\draw [thick] (5.2, 0) rectangle (5.7, 1);
				\draw [fill = black] (5.85, 0.475) circle (0.5mm);
				
				\draw [thick] (6.25, -0.2) arc (-90:-270:0.25 and 0.7);
				
				\draw [fill = black] (6.35, 0.04) circle (0.5mm);
				\draw [fill = black] (6.5, 0.04) circle (0.5mm);
				\draw [fill = black] (6.65, 0.04) circle (0.5mm);
				
				\draw [thick] (6.8, -0.2) arc (-90:90:0.25 and 0.7);
				
				\draw [thick] (5.1, -0.2) arc (180:360:1 and 0.3);
				\coordinate [label = below:Мн-во цепочек.] (B) at (6.2, -0.4);
				\coordinate [label = below:В начале] (C) at (6.2, -0.9);
				\coordinate [label = below:вертикальная] (C1) at (6.2, -1.475);
				
				% Плюс и далее
				\draw [thick] (7.7, 0.325) -- (7.7, 0.675);
				\draw [thick] (7.5, 0.5) -- (7.9, 0.5);
				
				\draw [thick] (8.3, 1) rectangle (9.3, 0.5);
				\draw [thick] (8.3, 0.5) rectangle (9.3, 0);
				
				\draw [fill = black] (9.45, 0.475) circle (0.5mm);
				
				\draw [thick] (9.85, -0.2) arc (-90:-270:0.25 and 0.7);
				
				\draw [fill = black] (9.95, 0.04) circle (0.5mm);
				\draw [fill = black] (10.1, 0.04) circle (0.5mm);
				\draw [fill = black] (10.25, 0.04) circle (0.5mm);
				
				\draw [thick] (10.35, -0.2) arc (-90:90:0.25 and 0.7);
				
				\draw [thick] (8.2, -0.2) arc (180:360:1.2 and 0.3);
				\coordinate [label = below:Мн-во всех] (D) at (9.5, -0.4);
				\coordinate [label = below:цепочек. В нач.] (E) at (9.5, -0.9);
				\coordinate [label = below:2 гориз.] (E1) at (9.5, -1.4);
			}
			\caption{\small Множество всех цепочек}
		\end{figure}
	
		Тут есть один фокус, может не все его заметили:
		
		\begin{figure}[H]
			\centering
			\tikz{
				\draw [thick] (0, 0) rectangle (0.5, 1);
			}
		\end{figure}
		
		Вот это цепочка или не цепочка? Цепочка, и цепочкой могут быть две горизонтальные доминошки. Если в из такой цепочки вынести одну вертикальную доминошку, то вместо неё ничего не останется. Но, если просто ничего не писать, это приведет к потере этой комбинации. Значит мне нужно ввести какой-то символ, к примеру, $\Lambda$, который будет означать пустоту.
		
		\[
			\Lambda \text{~--- пустая цепочка}
		\]
		
		То есть это не пустое множество, а именно цепочка, к которой можно присоединить другое домино, и получится цепочка из одного домино, поэтому на самом деле в первую и вторую скобки нужно вставить вот такой символ, тогда у меня действительно все цепочки будут учтены:
		
		\begin{figure}[H]
			\centering
			\tikz{
				\coordinate [label = below:Мн-во всех] (A) at (2.4, 0.9);
				\coordinate [label = below:домино-цепочек] (A1) at (2.4, 0.4);
				
				\draw [thick] (4.1, 0.525) -- (4.45, 0.525);
				\draw [thick] (4.1, 0.475) -- (4.45, 0.475);
				
				\draw [thick] (4.7, 0) rectangle (5.2, 1);
				\draw [fill = black] (5.35, 0.475) circle (0.5mm);
				
				\draw [thick] (5.75, -0.2) arc (-90:-270:0.25 and 0.7);
				
				\draw [thick, <-] (5.9, 0.7) -- (6.1, 1.3);
				\coordinate [label = above:$\Lambda$ + ...] (E) at (6.2, 1.25);
				
				\draw [fill = black] (5.95, 0.04) circle (0.5mm);
				\draw [fill = black] (6.1, 0.04) circle (0.5mm);
				\draw [fill = black] (6.25, 0.04) circle (0.5mm);
				
				\draw [thick] (6.4, -0.2) arc (-90:90:0.25 and 0.7);
				
				\draw [thick] (4.6, -0.2) arc(180:360:1.05 and 0.3);
				\coordinate [label = below:Мн-во цепочек.] (B) at (5.7, -0.4);
				\coordinate [label = below:В нач.] (C) at (5.7, -0.9);
				\coordinate [label = below:вертикальная] (C1) at (5.7, -1.5);  
				
				% Плюс и далее
				\draw [thick] (7.3, 0.325) -- (7.3, 0.675);
				\draw [thick] (7.125, 0.5) -- (7.475, 0.5);
				
				\draw [thick] (8, 1) rectangle (9, 0.5);
				\draw [thick] (8, 0.5) rectangle (9, 0);
				
				\draw [fill = black] (9.15, 0.475) circle (0.5mm);
				
				\draw [thick] (9.6, -0.2) arc (-90:-270:0.25 and 0.7);
				
				\draw [thick, <-] (9.6, 0.7) -- (9.8, 1.3);
				\coordinate [label = above:$\Lambda$ + ...] (E) at (9.9, 1.25);
				
				\draw [fill = black] (9.75, 0.04) circle (0.5mm);
				\draw [fill = black] (9.9, 0.04) circle (0.5mm);
				\draw [fill = black] (10.05, 0.04) circle (0.5mm);
				
				\draw [thick] (10.2, -0.2) arc (-90:90:0.25 and 0.7);
				
				\draw [thick] (7.9, -0.2) arc (180:360:1.25 and 0.3);
				\coordinate [label = below:Мн-во цепочек.] (D) at (9.3, -0.4);
				\coordinate [label = below:В нач. 2 гориз.] (E) at (9.3, -0.825);
			}
		\end{figure}
		
		Кроме того, в левой части уравнения из домино пустая цепочка есть, а в правой пустой цепочки нет, поэтому и тут нужно добавить эту пустую цепочку:
		
		\begin{figure}[H]
			\centering
			\tikz{
				\coordinate [label = below:Мн-во всех] (A) at (2.4, 0.9);
				\coordinate [label = below:домино-цепочек] (A1) at (2.4, 0.4);
				
				\draw [thick] (4.1, 0.525) -- (4.45, 0.525);
				\draw [thick] (4.1, 0.475) -- (4.45, 0.475);
				
				\draw [thick, <-] (4.535, 0.7) -- (4.635, 1.3);
				\coordinate [label = above:$\Lambda$ +] (G) at (4.7, 1.25);
				
				\draw [thick] (4.7, 0) rectangle (5.2, 1);
				\draw [fill = black] (5.35, 0.475) circle (0.5mm);
				
				\draw [thick] (5.75, -0.2) arc (-90:-270:0.25 and 0.7);
				
				\draw [thick, <-] (5.9, 0.7) -- (6.1, 1.3);
				\coordinate [label = above:$\Lambda$ + ...] (E) at (6.2, 1.25);
				
				\draw [fill = black] (5.95, 0.04) circle (0.5mm);
				\draw [fill = black] (6.1, 0.04) circle (0.5mm);
				\draw [fill = black] (6.25, 0.04) circle (0.5mm);
				
				\draw [thick] (6.4, -0.2) arc (-90:90:0.25 and 0.7);
				
				\draw [thick] (4.6, -0.2) arc(180:360:1.05 and 0.3);
				\coordinate [label = below:Мн-во цепочек.] (B) at (5.7, -0.4);
				\coordinate [label = below:В нач.] (C) at (5.7, -0.9);
				\coordinate [label = below:вертикальная] (C1) at (5.7, -1.5);  
				
				% Плюс и далее
				\draw [thick] (7.3, 0.325) -- (7.3, 0.675);
				\draw [thick] (7.125, 0.5) -- (7.475, 0.5);
				
				\draw [thick] (8, 1) rectangle (9, 0.5);
				\draw [thick] (8, 0.5) rectangle (9, 0);
				
				\draw [fill = black] (9.15, 0.475) circle (0.5mm);
				
				\draw [thick] (9.6, -0.2) arc (-90:-270:0.25 and 0.7);
				
				\draw [thick, <-] (9.6, 0.7) -- (9.8, 1.3);
				\coordinate [label = above:$\Lambda$ + ...] (E) at (9.9, 1.25);
				
				\draw [fill = black] (9.75, 0.04) circle (0.5mm);
				\draw [fill = black] (9.9, 0.04) circle (0.5mm);
				\draw [fill = black] (10.05, 0.04) circle (0.5mm);
				
				\draw [thick] (10.2, -0.2) arc (-90:90:0.25 and 0.7);
				
				\draw [thick] (7.9, -0.2) arc (180:360:1.25 and 0.3);
				\coordinate [label = below:Мн-во цепочек.] (D) at (9.3, -0.4);
				\coordinate [label = below:В нач. 2 гориз.] (E) at (9.3, -0.825);
			}
		\end{figure}
		
		Вот получается такая интересная арифметика. Значит еще раз, что в этой арифметике есть: плюс~--- это объединение. Я перечисляю все комбинации, и, если пишу плюсик, значит эти выражения рассматриваются вместе, умножить~--- это склеивание двух цепочек. Вот в следующем семестре опять же у вас будут регулярные выражения так вот это на самом деле регулярные выражения для описания множеств.
		
		\timestamp{65:58}
		
		Ну а дальше, чтобы, может быть, было привычней, я уже уйду от языка домино и напишу это в более привычном алгебраическом виде. Вместо вертикальной доминошки мы будем писать $x$, вместо двух горизонтальных~--- $y$, а вместо $\Lambda$ я буду писать 1. Понятно, почему единицу. Потому что при приклеивании, если бы я его назвал нулем, естественно 0 умножить на что-то будет 0, а 1 умножить на что-то будет что-то. Поэтому тут она больше похожа на единицу своим действием, чем на ноль. Тогда у меня получается, если я все множество цепочек обозначу за $S$, то у меня получится интересное равенство:
		
		\[
			S = 1 + x(...)
		\]
		
		А потом у меня в этой скобке опять идет то же самое множество. Похоже на периодические непрерывные дроби, когда одна конструкция была подконструкцией ее же самой:
		
		\[
		S = 1 + x \cdot S + y \cdot S
		\]
		
		Если я хочу перенести все с $S$ в одну часть, то придется ввести минусы, хотя будет непонятно что значит минус, что значит вычитать домино. Попробуем выполнить алгебраические действия, временно оторвавшись от их смысла:
		
		\[
			S \cdot (1 - x - y) = 1 \implies S = \frac{1}{1 - x - y}
		\]
		
		Если я это перепишу в терминах домино, получится очень забавно:
		
		\begin{figure}[H]
			\centering
			\tikz{
				\coordinate [label=left:$\Lambda$+] (A) at (0, 0.45);
				\draw [thick] (0, 0) rectangle (0.5, 1);
				
				\coordinate [label=right:+] (B) at (0.5, 0.45);
				\draw [thick] (1.1, 1) rectangle (2.1, 0.5);
				\draw [thick] (1.1, 0.5) rectangle (2.1, 0);
				
				\coordinate [label=right:+] (C) at (2.1, 0.45);
				\draw [thick] (2.7, 0) rectangle (3.2, 1);
				\draw [thick] (3.2, 1) rectangle (4.2, 0.5);
				\draw [thick] (3.2, 0.5) rectangle (4.2, 0);
				
				\coordinate [label=right:+...] (D) at (4.2, 0.45);
				
				\draw (5.1, 0.4) -- (5.4, 0.4);
				\draw (5.1, 0.5) -- (5.4, 0.5);
				\draw [thick] (5.5, 0.45) -- (8.1, 0.45);
				
				\coordinate [label=above:$\Lambda$] (E) at (6.45, 0.45);
				\coordinate [label=below:$\Lambda - $] (F) at (5.83, 0.32);
				\draw [thick] (6.2, -0.4) rectangle (6.7, 0.4);
				\draw (6.8, 0) -- (7.1, 0);
				\draw [thick] (7.25, 0.4) rectangle (8.05, 0);
				\draw [thick] (7.25, 0) rectangle (8.05, -0.4);
			}
		\end{figure}
	
		Вот такая абстрактная игра с символами, но оказывается она очень полезная. Когда я пишу $\frac{1}{1 - x - y}$, это вас не удивляет и скорее всего не удивляет, потому что вы знаете, что такое сумма бесконечно убывающей геометрической прогрессии. Если вы напишете это вот так:
		
		\[
			\frac{1}{1 - (x + y)} = 1 + x + y + (x + y)^2 + (x + y)^3 + \ldots
		\]
		
		Но опять же в терминах домино получается, что множество всех цепочек, я переписал вот в таком виде:
		
		\begin{figure}[H]
			\centering
			\tikz{
				\coordinate [label=left:$\Lambda$+] (A) at (0, 0.45);
				\draw [thick] (0, 0) rectangle (0.5, 1);
				
				\coordinate [label=right:+] (B) at (0.5, 0.45);
				\draw [thick] (1.1, 1) rectangle (2.1, 0.5);
				\draw [thick] (1.1, 0.5) rectangle (2.1, 0);
				\coordinate [label=right:+] (C) at (2.1, 0.45);
				
				\draw [thick] (2.95, -0.2) arc (-90:-270:0.25 and 0.7);
				\draw [thick] (3.05, 0) rectangle (3.55, 1);
				\coordinate [label=right:+] (B) at (3.55, 0.45);
				\draw [thick] (4.15, 1) rectangle (5.15, 0.5);
				\draw [thick] (4.15, 0.5) rectangle (5.15, 0);
				\draw [thick] (5.25, -0.2) arc (-90:90:0.25 and 0.7);
				\coordinate [label=right:2] (D) at (5.4, 1);
				\coordinate [label=right:+...] (E) at (5.4, 0.45);
			}
		\end{figure}
	
		А что за выражение в квадрате здесь уже можно понять: умножение скобки на себя, при этом, например, умножение вертикального элемента на горизонтальные будет цепочкой из двух таких частей:
		
		\begin{figure}[H]
			\centering
			\tikz{
				\draw [thick] (2.7, 0) rectangle (3.2, 1);
				\draw [thick] (3.2, 1) rectangle (4.2, 0.5);
				\draw [thick] (3.2, 0.5) rectangle (4.2, 0);
			}
		\end{figure}
		
		Получается, что мы множество всех цепочек перечислили, переструктурировав порядок. Так вот эта идея называется <<Производящая функция>>. Мы с ней будем целую лекцию возиться, потому что она одна из самых употребляемых для решения задач комбинаторики в научных кругах. Вместо того чтобы работать с комбинаторными объектами, будем работать с алгебраическими.
		
		Последнее, что мы сегодня пройдём~--- метод включений-исключений.
		
		\timestamp{70:43}
		
		\item \textbf{Принцип включения-исключения}. Inclusion-exclusion principle, так он по-английски звучит. Мне очень удобно начать с задачки, но эту задачку мы сочиним вместе.
		
		\timestamp{71:51}
		
		\begin{newProblem}
			Я нарисую так называемую диаграмму Эйлера, может быть, вы уже с ней сталкивались: 
			
			\begin{figure}[H]
				\centering
				\tikz{
					\draw [thick] (-3, -2) rectangle (3, 2);
					
					\draw [thick] (-0.75, 0.5) circle (1cm);
					\draw [thick] (0.75, 0.5) circle (1cm);
					\draw [thick] (0, -0.5) circle (1cm);
				}
			\end{figure}
			
			Круги~--- это подмножества, прямоугольник~--- это объемлющее множество. В нем есть такие подмножества: $A$~--- это будут те, кто изучает английский язык, допустим, F~--- французский язык, N~--- немецкий:
			
			\begin{figure}[H]
				\centering
				\tikz{
					\draw [thick] (-3, -2) rectangle (3, 2);
					
					\draw [thick] (-0.75, 0.5) circle (1cm);
					\draw [thick] (0.75, 0.5) circle (1cm);
					\draw [thick] (0, -0.5) circle (1cm);
					
					\coordinate [label=left:$A$] (A) at (-1.3, 1.4);
					\coordinate [label=right:$F$] (F) at (1.3, 1.4);
					\coordinate [label=below:$N$] (N) at (0, -1.5);
				}
			\end{figure}
			
			Тогда понятно, что множество за пределами кругов те, кто вообще этих языков не изучает, на пересечении всех множеств будут те, кто изучает три языка, ну, и так далее. Давайте придумаем конкретные значения некоторых параметров. Допустим есть всего один человек, который изучает три языка, три человека изучают немецкий и французский, два человека изучают английский и немецкий, четыре человека изучают английский и французский, сто человек изучают только английский, четыре человека изучают только французский, пять человек изучают только немецкий и 6 человек изучают китайский (то есть не изучают эти три языка):
			
			\begin{figure}[H]
				\centering
				\tikz{
					\draw [thick] (-3, -2) rectangle (3, 2);
					
					\draw [thick] (-0.75, 0.5) circle (1cm);
					\draw [thick] (0.75, 0.5) circle (1cm);
					\draw [thick] (0, -0.5) circle (1cm);
					
					\coordinate [label=left:$A$] (A) at (-1.3, 1.4);
					\coordinate [label=right:$F$] (F) at (1.3, 1.4);
					\coordinate [label=below:$N$] (N) at (0, -1.5);
					
					\coordinate [label=above:1] (B) at (0, -0.05);
					\coordinate [label=above:3] (B) at (0, 0.5);
					\coordinate [label=above:5] (B) at (0, -1);
					\coordinate [label=above:100] (B) at (-0.9, 0.4);
					\coordinate [label=above:4] (B) at (0.9, 0.4);
					\coordinate [label=above:1] (B) at (-0.45, -0.35);
					\coordinate [label=above:2] (B) at (0.45, -0.35);
					\coordinate [label=above:6] (B) at (2, -1.5);
				}
			\end{figure}
		
			Ну вот, а теперь давайте посчитаем сколько у нас всего человек. Пусть $M$~--- общее число человек. Сколько всего человек то получилось... 122 что ли? $M = 122$. Теперь задачка будет такая: английский язык изучает $M_A$ человек: $M_A = 105$. Французский язык изучают 10 человек: $M_F = 10$, немецкий язык изучают 9 человек: $M_N = 9$, английский и французский изучают 4 человека $M_{AF} = 4$, английский и немецкий изучают 2 человека $M_{AN} = 2$, французский и немецкий изучают 3 человека $M_{FN} = 3$. Ну и наконец все три языка изучает 1 человек: $M_{AFN} = 1$. И вот мне нужно написать формулу, она и будет называться формулой включений-исключений, для нахождения $M_0$~--- сколько человек не изучает ни один из этих трех языков.
			
		\end{newProblem}
	
		\timestamp{75:59}
	
		\begin{solution}
			Я могу рассуждать так: взять всех людей и вычесть тех, кто изучает английский, тех, кто изучает французский, и тех, кто изучает немецкий:
			
			\[
				M(0) = M - M_A - M_F - M_N + \ldots
			\]
			
			Правильно я поступлю? Не очень, потому что я некоторых исключу по два раза, а одного человека, который изучает 3 языка, я даже трижды исключу. Давайте сначала компенсируем тех, кого я дважды вычитаю, кто знает два языка, их по два раза вычел. Допустим изучающих английский и немецкий вычел и в $M_A$, и в $M_N$, значит я его должен прибавить:
			
			\[
				M(0) = M - M_A - M_F - M_N + M_{AF} + M_{AN} + M_{FN} + \ldots
			\]
			
			Теперь все нормально или нет? Теперь с теми, кто знает ровно два языка, всё хорошо: я их сначала два раза вычел, а потом один раз добавил. Но человека, который знает 3 языка, я сначала три раза убрал, а потом три раза добавил. Что нам теперь сделать? Один раз вычесть, потому что мне его считать не надо:
			
			\[
				M(0) = M - M_A - M_F - M_N + M_{AF} + M_{AN} + M_{FN} - M_{AFN}
			\]
			
			Вот это получилось формула включений-исключений для трех элементов. Теперь давайте ее обобщим на общий случай.
			
		\end{solution}
	
		Теперь мы будем говорить о признаках. Допустим у нас есть $M_1$ элементов, обладающих первым признаком:
		
		\[
			M_1 \text{~--- число элементов, обладающих первым признаком}
		\]
		
		Понятно что тут опять два языка: один язык логический, с которым мы будем с вами иметь дело в следующем семестре, а другой~--- язык теории множеств, чем мы сейчас занимаемся. Так вот можно говорить, что элемент принадлежит к множеству А. Это я в терминах языка теории множеств говорю, а могу говорить, что он обладает признаком А, то есть знает английский язык. Это уже логическое: знает или не знает, да или нет.
		
		\[
			M_2 \text{~--- число элементов, обладающих вторым признаком}
		\]
		\[
			\vdots
		\]
		\[
			M_n \text{~--- число элементов, обладающих $n$-ым признаком}
		\]
		
		Теперь я могу рассматривать их комбинации, но я уже напишу в общем случае. $M_{i_1i_2...i_k}$~--- это как бы пересечение множеств, соответствующих $M_{i_1}$, $M_{i_2}$, и так далее до $M_{i_k}$, а в терминах свойств~--- это будет число элементов, обладающих признаками с этими номерами (признаками $i_1$, $i_2$, $\ldots$, $i_k$). Теперь давайте попробуем написать общую формулу, она и будет общей формулой включений-исключений. Значит количество элементов, не обладающих ни одним из признаков, будет вычисляться так. Сначала вычитаются все элементы, обладающие одним признаком:
		
		\[
			M(0) = M - \sum_{i=1}^{n}M_i + \ldots
		\]
		
		Но потом по тем причинам, что я слишком много вычел, я добавлю все попарные комбинации признаков. И вот здесь обратите внимание, когда я добавлял в предыдущем примере попарные комбинации, я же не добавлял отдельно тех, кто знает английский и французский к тем, кто знает французский и английский. Поэтому мне эти признаки надо упорядочить и использовать каждый набор признаков только один раз. Поэтому признаки берут в порядке возрастания номеров, и тогда часто просто пишут $i < j$, предполагая, что значение каждого индекса находится в пределах от 1 до n:
		
		\[
			M(0) = M - \sum_{i=1}^{n}M_i + \sum_{i<j}M_{ij} \ldots
		\]
		
		Но и понятно, что в общем случае знаки чередуются. У меня будет $-1$ в степени $k$, наверное, да? Потом проверим:
		
		\[
			M(0) = M - \sum_{i=1}^{n}M_i + \sum_{i<j}M_{ij} \ldots (-1)^k \sum_{i_1<\ldots<i_k}M_{i_1 \ldots i_k} \ldots
		\]
		
		Все эти признаки идут в порядке возрастания. Ну, давайте проверим, если $k = 1$, то получается минус, если $k = 2$, то плюс. Значит все правильно со знаками. И как выглядит последний элемент? Он на самом деле всего один, потому что существует всего одна ситуация, когда все признаки от 1 до n есть, но в зависимости от четности, нечетности этот член может оказаться с минусом или с плюсом:
		
		\[
			M(0) = M - \sum_{i=1}^{n}M_i + \sum_{i<j}M_{ij} \ldots (-1)^k \sum_{i_1<\ldots<i_k}M_{i_1 \ldots i_k} \ldots (-1)^n M_{123 \ldots n}
		\]
		
		Поскольку у меня время заканчивается, я хочу задать вам одну задачку на метод включений-исключений	дополнительно, кто хочет, тот решит и покажет мне, выложив на свой сайт.
		
		\timestamp{82:45}
		
		\begin{hats}
			$N$ джентльменов в клубе оставили в гардеробе $N$ шляп, а при выходе каждый надел случайную шляпу. Какова вероятность, что никто не надел свою шляпу?
		\end{hats}
	
		Ну, раз появилось слово вероятность, я должен связать это с термином комбинаторика. Ну все просто, если эти джентльмены могли одеть шляпу $N$ способами и среди них $N_{\text{благоприятных}}$, вряд ли для мужчин благоприятно уйти не в свой шляпе, но с точки зрения комбинаторики это те варианты, когда никто из них не в своей шляпе. Далее вы делите одно на другое, получается вероятность того, что никто из них не ушел в своей шляпе:
		
		\[
			P = \frac{N_{\text{благоприятных}}}{N}
		\]
		
	\end{enumerate}
	
\end{document}
